% interactcadsample.tex
% v1.03 - April 2017

\documentclass[]{interact}

\usepackage{epstopdf}% To incorporate .eps illustrations using PDFLaTeX, etc.
\usepackage{subfigure}% Support for small, `sub' figures and tables
%\usepackage[nolists,tablesfirst]{endfloat}% To `separate' figures and tables from text if required

\usepackage{natbib}% Citation support using natbib.sty
\bibpunct[, ]{(}{)}{;}{a}{}{,}% Citation support using natbib.sty
\renewcommand\bibfont{\fontsize{10}{12}\selectfont}% Bibliography support using natbib.sty

\theoremstyle{plain}% Theorem-like structures provided by amsthm.sty
\newtheorem{theorem}{Theorem}[section]
\newtheorem{lemma}[theorem]{Lemma}
\newtheorem{corollary}[theorem]{Corollary}
\newtheorem{proposition}[theorem]{Proposition}

\theoremstyle{definition}
\newtheorem{definition}[theorem]{Definition}
\newtheorem{example}[theorem]{Example}

\theoremstyle{remark}
\newtheorem{remark}{Remark}
\newtheorem{notation}{Notation}

% see https://stackoverflow.com/a/47122900

% Pandoc citation processing

\usepackage{hyperref}
\usepackage[utf8]{inputenc}
\def\tightlist{}


\begin{document}

\articletype{ARTICLE TEMPLATE}

\title{Why shouldn't you use numerical tests to diagnose the linear
regression models?}


\author{\name{Weihao Li$^{a}$, Dianne Cook$^{a}$, Emi Tanaka$^{a}$}
\affil{$^{a}$Department of Econometrics and Business Statistics, Monash
University, Clayton, VIC, Australia}
}

\thanks{CONTACT Weihao
Li. Email: \href{mailto:weihao.li@monash.edu}{\nolinkurl{weihao.li@monash.edu}}, Dianne
Cook. Email: \href{mailto:dicook@monash.edu}{\nolinkurl{dicook@monash.edu}}, Emi
Tanaka. Email: \href{mailto:emi.tanaka@monash.edu}{\nolinkurl{emi.tanaka@monash.edu}}}

\maketitle

\begin{abstract}
Abstract to fill.
\end{abstract}

\begin{keywords}
visual inference; model diagnostics;
\end{keywords}

problem: residual plot diagnostics conventional test: too sensitive

background:

\begin{enumerate}
\def\labelenumi{\arabic{enumi}.}
\tightlist
\item
  residual plot for model diagnostics
\end{enumerate}

\begin{enumerate}
\def\labelenumi{\alph{enumi}.}
\tightlist
\item
  residual is widely used
\item
  what are the types of residual plots
\item
  comparison
\end{enumerate}

\begin{enumerate}
\def\labelenumi{\arabic{enumi}.}
\setcounter{enumi}{1}
\tightlist
\item
  conventional test: F, BP
\item
  visual test: lineup, theory
\end{enumerate}

desc of experiment: 1. simulation setup 2. experimental design 3. result

comparison of conventional tests: 1. power (visual test vs.~conventional
test) (visual test most different one (everything test, any departure))
2. investigate the difference (gap), give examples 3. conventional is
too sensitive 4. make conventional less sensitive (vary alpha)

conclusion: 1. too sensitive, visual test is needed/preferable 2. visual
test is infeasible in large scale (expensive) 3. future work (role of
computer vision)

\hypertarget{introduction}{%
\section{Introduction}\label{introduction}}

Regression diagnostics conventionally involve evaluating the fitness of
the proposed model, detecting the presence of influential observations
and outliers, checking the validity of model assumptions and many more.
Common diagnostic techniques including summary statistics, hypothesis
testing, and data plots are essential tools for a systematic and
detailed examination of the regression model
\citep{mansfield1987diagnostic}.

\hypertarget{diagnostic-plots}{%
\subsection{Diagnostic plots}\label{diagnostic-plots}}

Regression analysis is a field of study with at least a hundred years of
history. Many of those regression diagnostic methods and procedures are
mature and well-established in books first published in the twentieth
century, such as \citet{draper_applied_2014},
\citet{montgomery_introduction_2012}, \citet{belsley_regression_1980},
\citet{cook_applied_1999} and \citet{cook1982residuals}. Regardless of
the level of difficulty of the book, one will find the importance and
usefulness of diagnostic plots being emphasized again and again.
Checking diagnostic plots is also the recommended starting point for
validating model assumptions like normality, homoscedasticity and
linearity \citep{anscombe_examination_1963}.

Graphical summaries in which residuals are plotted against fitted values
or other functions of the predictor variables that are approximately
orthogonal to residuals are refereed to as standard residual plots. They
are commonly used to identify patterns which are indicative of
nonconstant error variance or non-linearity \citep{cook1982residuals}.
Raw residuals and studentized residuals are the two most frequently used
residuals in standard residual plots. The debt on which type of
residuals should be used always present. While raw residuals are the
most common output of computer regression software package, by applying
a scaling factor, the ability of revealing nonconstant error variance in
standard residual plots will often be enhanced by studentized residuals
in small sample size \citep{gunst2018regression}.

As a two-dimensional representation of a model in a \(p\)-dimensional
space, standard residual plots project data points onto the variable of
the horizontal axis, which is a vector in \(p\)-dimensional space.
Observations with the same projection will be treated as equivalent as
they have the same position of the abscissa. Therefore, standard
residual plots are often useful in revealing model inadequacies in the
direction of the variable of the horizontal axis, but could be
inadequate for detecting patterns in other directions, especially in
those perpendicular to the variable of the horizontal axis. Hence, in
practice, multiple standard residual plots with different horizontal
axes will be examined.

Overlapping data points is a general issue in scatter plots not limited
to standard residual plots, which often makes plots difficult to
interpret because visual patterns are concealed. Thus, for relatively
large sample size, \citet{cleveland1975graphical} suggests the use of
robust moving statistics as reference lines to give aids to eye in
seeing patterns, which nowadays, are usually replaced with a spline or
local polynomial regression line.

Other types of data plots that are often used in regression diagnostics
include partial residual plots and probability plots. Partial residual
plots are useful supplements to standard residual plots as they provide
additional information on the extent of the non-linearity. Probability
plots can be used to compare the sampling distribution of the residuals
to the normal distribution for assessing the normality assumptions.

\hypertarget{hypothesis-testing}{%
\subsection{Hypothesis testing}\label{hypothesis-testing}}

In addition to diagnostic plots, researcher may also perform formal
tests for detecting model defects. Depends on the alternative, variety
of tests can be applied. For example, for testing heteroskedasticity,
one may use the White test
\citep{white_heteroskedasticity-consistent_1980} or the Breusch-Pagan
test \citep{breusch_simple_1979}. And for testing non-linearity, there
are RESET test \citep{ramsey_tests_1969} and F-test.

As discussed in \citet{cook1982residuals}, most residual based tests for
a particular type of departures from model assumptions are sensitive to
other types of departures. Especially, outliers will often incorrectly
trigger the rejection of null hypothesis despite the residuals are
well-behaved \citep{cook_applied_1999}. This can be largely avoided in
diagnostic plots as experienced analysts can evaluate the acceptability
of assumptions flexibly, even in the presence of outliers. Furthermore,
\citet{montgomery_introduction_2012} stated that based on their
experience, statistical tests are not widely used in regression
diagnostics. Most importantly, the same or even larger amount of
information can be provided by diagnostic plots than the corresponding
tests in most empirical studies. But still, the effectiveness of
statistical tests shall not be disrespected. Statistical tests have
chance to provide analysts with unique information. There are also
situations where no suitable diagnostic plots can be found for a
particular violation of the assumptions, or excessive diagnostic plots
need to be checked. One will have no choice but to rely on statistical
tests if there is any. A good regression diagnostic practice should be a
combination of both methods.

\hypertarget{visual-inference}{%
\subsection{Visual inference}\label{visual-inference}}

Diagnostic plots are

However, unlike confirmatory data analysis built upon rigorous
statistical procedures, e.g., hypothesis testing, visual diagnostics
relies on graphical perception - human's ability to interpret and decode
the information embedded in the graph \citep{cleveland_graphical_1984},
which is to some extent subjective. Further, visual discovery suffers
from its unsecured and unconfirmed nature where the degree of the
presence of the visual features typically can not be measured
quantitatively and objectively, which may lead to over or
under-interpretations of the data. One such example is finding an
over-interpretation of the separation between gene groups in a
two-dimensional projection from a linear discriminant analysis when in
fact there are no differences in the expression levels between the gene
groups and separation is not an uncommon occurrence
\citep{roy_chowdhury_using_2015}.

Recently, a new branch of statical inference

Visual inference was first introduced in a 1999 Joint Statistical
Meetings (JSM) talk with the title ``Inference for Data Visualization''
by \citet{buja_inference_1999} as an idea to address the issue of valid
inference for visual discoveries of data plots
\citep{gelman_exploratory_2004}. Later, in the Bayesian context, data
plots was systematically considered as model diagnostics by taking
advantage of the data simulated from the assumed statistical models
\citep{gelman_bayesian_2003, gelman_exploratory_2004}.

\bibliographystyle{tfcad}
\bibliography{paper.bib}




\end{document}
