% interactcadsample.tex
% v1.03 - April 2017

\documentclass[]{interact}

\usepackage{epstopdf}% To incorporate .eps illustrations using PDFLaTeX, etc.
\usepackage{subfigure}% Support for small, `sub' figures and tables
%\usepackage[nolists,tablesfirst]{endfloat}% To `separate' figures and tables from text if required

\usepackage{natbib}% Citation support using natbib.sty
\bibpunct[, ]{(}{)}{;}{a}{}{,}% Citation support using natbib.sty
\renewcommand\bibfont{\fontsize{10}{12}\selectfont}% Bibliography support using natbib.sty

\theoremstyle{plain}% Theorem-like structures provided by amsthm.sty
\newtheorem{theorem}{Theorem}[section]
\newtheorem{lemma}[theorem]{Lemma}
\newtheorem{corollary}[theorem]{Corollary}
\newtheorem{proposition}[theorem]{Proposition}

\theoremstyle{definition}
\newtheorem{definition}[theorem]{Definition}
\newtheorem{example}[theorem]{Example}

\theoremstyle{remark}
\newtheorem{remark}{Remark}
\newtheorem{notation}{Notation}

% see https://stackoverflow.com/a/47122900

% Pandoc citation processing

\usepackage{hyperref}
\usepackage[utf8]{inputenc}
\def\tightlist{}


\begin{document}

\articletype{ARTICLE TEMPLATE}

\title{Why shouldn't you use numerical tests to diagnose the linear
regression models?}


\author{\name{Weihao Li$^{a}$, Dianne Cook$^{a}$, Emi Tanaka$^{a}$}
\affil{$^{a}$Department of Econometrics and Business Statistics, Monash
University, Clayton, VIC, Australia}
}

\thanks{CONTACT Weihao
Li. Email: \href{mailto:weihao.li@monash.edu}{\nolinkurl{weihao.li@monash.edu}}, Dianne
Cook. Email: \href{mailto:dicook@monash.edu}{\nolinkurl{dicook@monash.edu}}, Emi
Tanaka. Email: \href{mailto:emi.tanaka@monash.edu}{\nolinkurl{emi.tanaka@monash.edu}}}

\maketitle

\begin{abstract}
Abstract to fill.
\end{abstract}

\begin{keywords}
Sections; lists; figures; tables; mathematics; fonts; references;
appendices
\end{keywords}

problem: residual plot diagnostics conventional test: too sensitive

background:

\begin{enumerate}
\def\labelenumi{\arabic{enumi}.}
\tightlist
\item
  residual plot for model diagnostics
\end{enumerate}

\begin{enumerate}
\def\labelenumi{\alph{enumi}.}
\tightlist
\item
  residual is widely used
\item
  what are the types of residual plots
\item
  comparison
\end{enumerate}

\begin{enumerate}
\def\labelenumi{\arabic{enumi}.}
\setcounter{enumi}{1}
\tightlist
\item
  conventional test: F, BP
\item
  visual test: lineup, theory
\end{enumerate}

desc of experiment: 1. simulation setup 2. experimental design 3. result

comparison of conventional tests: 1. power (visual test vs.~conventional
test) (visual test most different one (everything test, any departure))
2. investigate the difference (gap), give examples 3. conventional is
too sensitive 4. make conventional less sensitive (vary alpha)

conclusion: 1. too sensitive, visual test is needed/preferable 2. visual
test is infeasible in large scale (expensive) 3. future work (role of
computer vision)

\hypertarget{introduction}{%
\section{Introduction}\label{introduction}}

Diagnostics of the classical normal linear regression model
conventionally involve evaluating the fitness of the proposed model,
detecting the presence of influential observations and outliers,
checking the validity of model assumptions and many more. Tools such as
summary statistics, hypothesis testing, and data plots are essential for
a systematic and detailed examination of the regression model
\citep{mansfield1987diagnostic}.

\hypertarget{data-plots}{%
\subsection{Data plots}\label{data-plots}}

Data plots are one of the most important and preferred methods of
regression diagnostics. Graphical summaries in which residuals are
plotted against fitted values or other functions of the predictor
variables that are approximately orthogonal to residuals are refereed to
as standard residual plots. They are commonly used to identify patterns
which are indicative of nonconstant error variance or nonlinearity
\citep{cook1982residuals}. Raw residuals and studentized residuals are
the two most frequently used residuals in standard residual plots. The
debt on which type of residuals should be used always present. While raw
residuals are the most common output of computer regression software
package, by applying a scaling factor, the ability of revealing
nonconstant error variance in standard residual plots will often be
enhanced by studentized residuals in small sample size
\citep{gunst2018regression}.

As a two-dimensional representation of a model in a \(p\)-dimensional
space, standard residual plots project data points onto the variable of
the horizontal axis, which is a vector in \(p\)-dimensional space.
Observations with the same projection will be treated as equivalent as
they have the same position of the abscissa. Therefore, standard
residual plots are often useful in revealing model inadequacies in the
direction of the variable of the horizontal axis, but could be
inadequate for detecting patterns in other directions, especially in
those perpendicular to the variable of the horizontal axis. Hence, in
practice, multiple standard residual plots with different horizontal
axes will be examined.

Overlapping data points is a general issue in scatter plots not limited
to standard residual plots, which often makes plots difficult to
interpret because visual patterns are concealed. Thus, for relatively
large sample size, \citet{cleveland1975graphical} suggests the use of
robust moving statistics as reference lines to give aids to eye in
seeing patterns, which nowadays, are usually replaced with a spline or
local polynomial regression line.

Other types of data plots that are often used in regression diagnostics
include partial residual plots and probability plots. Partial residual
plots are useful supplements to standard residual plots as they provide
additional information on the direction of linearity as well as the
nonlinearity of each predictor. Probability plots can be used to compare
the sampling distribution of the residuals to the normal distribution.

\hypertarget{hypothesis-testing}{%
\subsection{Hypothesis testing}\label{hypothesis-testing}}

\bibliographystyle{tfcad}
\bibliography{interactcadsample.bib}




\end{document}
