% Options for packages loaded elsewhere
\PassOptionsToPackage{unicode}{hyperref}
\PassOptionsToPackage{hyphens}{url}
%
\documentclass[
]{book}
\usepackage{amsmath,amssymb}
\usepackage{lmodern}
\usepackage{ifxetex,ifluatex}
\ifnum 0\ifxetex 1\fi\ifluatex 1\fi=0 % if pdftex
  \usepackage[T1]{fontenc}
  \usepackage[utf8]{inputenc}
  \usepackage{textcomp} % provide euro and other symbols
\else % if luatex or xetex
  \usepackage{unicode-math}
  \defaultfontfeatures{Scale=MatchLowercase}
  \defaultfontfeatures[\rmfamily]{Ligatures=TeX,Scale=1}
\fi
% Use upquote if available, for straight quotes in verbatim environments
\IfFileExists{upquote.sty}{\usepackage{upquote}}{}
\IfFileExists{microtype.sty}{% use microtype if available
  \usepackage[]{microtype}
  \UseMicrotypeSet[protrusion]{basicmath} % disable protrusion for tt fonts
}{}
\makeatletter
\@ifundefined{KOMAClassName}{% if non-KOMA class
  \IfFileExists{parskip.sty}{%
    \usepackage{parskip}
  }{% else
    \setlength{\parindent}{0pt}
    \setlength{\parskip}{6pt plus 2pt minus 1pt}}
}{% if KOMA class
  \KOMAoptions{parskip=half}}
\makeatother
\usepackage{xcolor}
\IfFileExists{xurl.sty}{\usepackage{xurl}}{} % add URL line breaks if available
\IfFileExists{bookmark.sty}{\usepackage{bookmark}}{\usepackage{hyperref}}
\hypersetup{
  pdftitle={PhD Notebook},
  pdfauthor={Patrick Li},
  hidelinks,
  pdfcreator={LaTeX via pandoc}}
\urlstyle{same} % disable monospaced font for URLs
\usepackage{longtable,booktabs,array}
\usepackage{calc} % for calculating minipage widths
% Correct order of tables after \paragraph or \subparagraph
\usepackage{etoolbox}
\makeatletter
\patchcmd\longtable{\par}{\if@noskipsec\mbox{}\fi\par}{}{}
\makeatother
% Allow footnotes in longtable head/foot
\IfFileExists{footnotehyper.sty}{\usepackage{footnotehyper}}{\usepackage{footnote}}
\makesavenoteenv{longtable}
\usepackage{graphicx}
\makeatletter
\def\maxwidth{\ifdim\Gin@nat@width>\linewidth\linewidth\else\Gin@nat@width\fi}
\def\maxheight{\ifdim\Gin@nat@height>\textheight\textheight\else\Gin@nat@height\fi}
\makeatother
% Scale images if necessary, so that they will not overflow the page
% margins by default, and it is still possible to overwrite the defaults
% using explicit options in \includegraphics[width, height, ...]{}
\setkeys{Gin}{width=\maxwidth,height=\maxheight,keepaspectratio}
% Set default figure placement to htbp
\makeatletter
\def\fps@figure{htbp}
\makeatother
\setlength{\emergencystretch}{3em} % prevent overfull lines
\providecommand{\tightlist}{%
  \setlength{\itemsep}{0pt}\setlength{\parskip}{0pt}}
\setcounter{secnumdepth}{5}
\usepackage{booktabs}
\ifluatex
  \usepackage{selnolig}  % disable illegal ligatures
\fi
\usepackage[]{natbib}
\bibliographystyle{apalike}

\title{PhD Notebook}
\author{Patrick Li}
\date{2021-06-20}

\begin{document}
\maketitle

{
\setcounter{tocdepth}{1}
\tableofcontents
}
\hypertarget{welcome}{%
\chapter{Welcome}\label{welcome}}

I am Patrick Li.

\hypertarget{introduction}{%
\chapter{Introduction}\label{introduction}}

This note consists of:

\begin{enumerate}
\def\labelenumi{\arabic{enumi}.}
\tightlist
\item
  records of weekly meetings
\item
  literature review
\item
  to-do list
\item
  milestones
\item
  links to resources
\end{enumerate}

\hypertarget{literature}{%
\chapter{Literature}\label{literature}}

\hypertarget{graphical-inference-for-infovis}{%
\section{Graphical Inference for Infovis}\label{graphical-inference-for-infovis}}

BibTex:

\begin{verbatim}
@article{wickham2010graphical,
  title={Graphical inference for infovis},
  author={Wickham, Hadley and Cook, Dianne and Hofmann, Heike and Buja, Andreas},
  journal={IEEE Transactions on Visualization and Computer Graphics},
  volume={16},
  number={6},
  pages={973--979},
  year={2010},
  publisher={IEEE}
}
\end{verbatim}

\hypertarget{keywords}{%
\subsection{Keywords}\label{keywords}}

Statistics, visual testing, permutation tests, null hypotheses, data plots.

\hypertarget{introduction-1}{%
\subsection{Introduction}\label{introduction-1}}

Infovis focuses on uncovering new relationships by \textbf{tools of curiosity}, but most statistical methods focuses on examining relationships by \textbf{tools of skepticism}. Neither extreme is good. Hence, graphical inference try to fill the gap between them. It claims that this kind of inference can provide a tool for skepticism that can be applied in a curiosity-driven context.

\hypertarget{what-is-inference-and-why-do-we-need-it}{%
\subsection{What is inference and why do we need it?}\label{what-is-inference-and-why-do-we-need-it}}

Inference is about drawing conclusions about the population from the sample. There are two components of statistical inference, testing and estimation. For graphical inference, the focus is to test whether what we see in a plot of the sample is an accurate reflection of the entire population or not. The test statistic in visual inference is a plot of the data. A \textbf{null dataset} is a sample from the null distribution, and a \textbf{null plot} is a plot of a null dataset. The benefit of visual inference is that it can be used in complex data analysis settings that do not have corresponding numerical tests.

\hypertarget{protocols-of-graphical-inference}{%
\subsection{Protocols of graphical inference}\label{protocols-of-graphical-inference}}

\hypertarget{rorschach}{%
\subsubsection{Rorschach}\label{rorschach}}

Rorschach protocol is used to calibrate our vision to the natural variability in plots in which the data is generated from scenarios consistent with the null hypothesis.

\hypertarget{line-up}{%
\subsubsection{Line-up}\label{line-up}}

The line up is consisted of \(n-1\) decoys and \(1\) plot of the true data. If we set \(n = 19\), then under the null hypothesis, there is only \(5\)\% chance to pick the plot of the true data. If we recruit \(K\) observers, then under the null hypothesis, the p-value is \(P(B(K, 0.05) \geq k)\).

\hypertarget{examples}{%
\subsection{Examples}\label{examples}}

To use the line-up protocol, we need to:
1. Identify the question the plot is trying to answer
2. Characterize the null-hypothesis
3. Figure out how to generate null datasets

There are two techniques that can be applied in many caeses:

\begin{enumerate}
\def\labelenumi{\arabic{enumi}.}
\tightlist
\item
  Resampling. Permutation and bootstrapping.
\item
  Simulated data from a assumed model.
\end{enumerate}

\hypertarget{tag-clouds}{%
\subsubsection{Tag clouds}\label{tag-clouds}}

A tag cloud can be used to visualize frequency of words in a document. Words are arranged in various ways, often alphabetically, with size proportional to their frequency. The null hypothesis for a comparison tag cloud is that the two documents are equivalent, the frequency of words is the same in each document. Null data can be generated by randomly permute the one of the column.

\hypertarget{scatterplot}{%
\subsubsection{Scatterplot}\label{scatterplot}}

A scatterplot displays the relationship between \(x\) and \(y\). A strong null hypothesis is that there is no relationship between \(x\) and \(y\) variables.

\hypertarget{power}{%
\subsection{Power}\label{power}}

The power of a statistical test is the probability of correctly convicting a guilty data set. The capacity to detect specific structure in plots can depend on the perceptual properties.

\hypertarget{use}{%
\subsection{Use}\label{use}}

An R package: nullabor

\hypertarget{meetings}{%
\chapter{Meetings}\label{meetings}}

\hypertarget{march-10-2021---week-2}{%
\section{March 10, 2021 - Week 2}\label{march-10-2021---week-2}}

\begin{enumerate}
\def\labelenumi{\arabic{enumi}.}
\tightlist
\item
  Human subject experiment and Monash permission

  \begin{itemize}
  \tightlist
  \item
    There is a workshop in May
  \end{itemize}
\item
  Use simulated data to set up an experiment
\item
  Check out Gallery of graphs (a name yan \ldots{} not sure)
\item
  The experiment could start from residual plot
\item
  Q-Q plot could also be considered
\item
  A revelant research - residual calculation by kaiwen - master project
\item
  Use Appen survey to collect data
\item
  Build a Github to-do list, meeting record and summary of the literature
\item
  There may be some development in the theory by Nancy Reid - theoretical statistician (recent work)
\item
  Consider using Kears to build a computer vision model
\end{enumerate}

\hypertarget{march-17-2021---week-3}{%
\section{March 17, 2021 - Week 3}\label{march-17-2021---week-3}}

\begin{enumerate}
\def\labelenumi{\arabic{enumi}.}
\tightlist
\item
  Read Susan Vanderplas's personal website to find additional information
\item
  Check out NUMBAT residual plot comparision - summer-vis-inf : Aarathy Babu - code examples
\item
  Read human subject premission examples (sent by Di)
\item
  Check out top-up application
\item
  Consider to use \texttt{Edibble} to set up the experiment
\item
  Build the PhD repo
\item
  Consider to use non-shiny framework
\end{enumerate}

\hypertarget{march-21-2021---week-4}{%
\section{March 21, 2021 - Week 4}\label{march-21-2021---week-4}}

\begin{enumerate}
\def\labelenumi{\arabic{enumi}.}
\tightlist
\item
  A short meeting - late for 30 minutes
\item
  Aarathy introduces her repo
\end{enumerate}

\hypertarget{march-28-2021---week-5}{%
\section{March 28, 2021 - Week 5}\label{march-28-2021---week-5}}

\begin{enumerate}
\def\labelenumi{\arabic{enumi}.}
\tightlist
\item
  Discuss the options of building a alternative hypothesis in residual plot
\item
  AR, heterogeneity of variance, endogeneity, skewness, exp distribution, poisson distribution and missing covariance
\item
  Choice of plot design (loess or \(y=0\) line), number of lineup (5\textasciitilde15), number of observations
\item
  Use flow chart to illustrate the choices
\item
  Literature review - check previous designs
\item
  Build bookdown to track records
\end{enumerate}

\hypertarget{todo}{%
\chapter{TODO}\label{todo}}

\hypertarget{week-3}{%
\section{Week 3}\label{week-3}}

\begin{itemize}
\tightlist
\item[$\boxtimes$]
  Check human subject experiment and Monash permission materials

  \begin{itemize}
  \tightlist
  \item
    \url{https://www.intranet.monash/researchadmin/start/ethics/human}
  \end{itemize}
\item[$\boxtimes$]
  Check Kaiwen's project \& paper
\item[$\square$]
  Check Gallery of graphs - unclear author
\end{itemize}

\hypertarget{week-4}{%
\section{Week 4}\label{week-4}}

\begin{itemize}
\tightlist
\item[$\boxtimes$]
  Build prototype of html webpage to collect data
\item[$\boxtimes$]
  Send data to google sheet
\item[$\boxtimes$]
  Check summer-vis-inf
\item[$\boxtimes$]
  Read examples sent by Di
\item[$\square$]
  Build PhD repo

  \begin{itemize}
  \tightlist
  \item[$\boxtimes$]
    meetings
  \item[$\square$]
    paper
  \item[$\boxtimes$]
    TODO
  \end{itemize}
\item[$\square$]
  Check Susan Vanderplas's website

  \begin{itemize}
  \tightlist
  \item[$\square$]
    paper
  \item[$\square$]
    talks
  \item[$\square$]
    posts
  \end{itemize}
\end{itemize}

\hypertarget{week-5}{%
\section{Week 5}\label{week-5}}

\begin{itemize}
\tightlist
\item[$\boxtimes$]
  modify the webpage to be able to select multiple plots
\item[$\boxtimes$]
  Attempt to generate data from one assumed model
\item[$\square$]
  draft Human Ethics Application Form
\end{itemize}

\hypertarget{week-6}{%
\section{Week 6}\label{week-6}}

\begin{itemize}
\tightlist
\item[$\square$]
  Do the literature review of previous design
\item[$\square$]
  Draw a flow chart to illustrate the design
\end{itemize}

\hypertarget{milestones}{%
\chapter{Milestones}\label{milestones}}

  \bibliography{book.bib,packages.bib}

\end{document}
